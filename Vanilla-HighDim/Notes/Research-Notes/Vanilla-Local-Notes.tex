\documentclass[11pt]{article}
%%% full alphabets of different styles %%%

% bf series
\def\bfA{\mathbf{A}}
\def\bfB{\mathbf{B}}
\def\bfC{\mathbf{C}}
\def\bfD{\mathbf{D}}
\def\bfE{\mathbf{E}}
\def\bfF{\mathbf{F}}
\def\bfG{\mathbf{G}}
\def\bfH{\mathbf{H}}
\def\bfI{\mathbf{I}}
\def\bfJ{\mathbf{J}}
\def\bfK{\mathbf{K}}
\def\bfL{\mathbf{L}}
\def\bfM{\mathbf{M}}
\def\bfN{\mathbf{N}}
\def\bfO{\mathbf{O}}
\def\bfP{\mathbf{P}}
\def\bfQ{\mathbf{Q}}
\def\bfR{\mathbf{R}}
\def\bfS{\mathbf{S}}
\def\bfT{\mathbf{T}}
\def\bfU{\mathbf{U}}
\def\bfV{\mathbf{V}}
\def\bfW{\mathbf{W}}
\def\bfX{\mathbf{X}}
\def\bfY{\mathbf{Y}}
\def\bfZ{\mathbf{Z}}
\def\bfx{\mathbf{x}}
\def\bfmu{\boldsymbol{\mu}}
\def\bfSigma{\mathbf{\Sigma}}

% bb series
\def\bbA{\mathbb{A}}
\def\bbB{\mathbb{B}}
\def\bbC{\mathbb{C}}
\def\bbD{\mathbb{D}}
\def\bbE{\mathbb{E}}
\def\bbF{\mathbb{F}}
\def\bbG{\mathbb{G}}
\def\bbH{\mathbb{H}}
\def\bbI{\mathbb{I}}
\def\bbJ{\mathbb{J}}
\def\bbK{\mathbb{K}}
\def\bbL{\mathbb{L}}
\def\bbM{\mathbb{M}}
\def\bbN{\mathbb{N}}
\def\bbO{\mathbb{O}}
\def\bbP{\mathbb{P}}
\def\bbQ{\mathbb{Q}}
\def\bbR{\mathbb{R}}
\def\bbS{\mathbb{S}}
\def\bbT{\mathbb{T}}
\def\bbU{\mathbb{U}}
\def\bbV{\mathbb{V}}
\def\bbW{\mathbb{W}}
\def\bbX{\mathbb{X}}
\def\bbY{\mathbb{Y}}
\def\bbZ{\mathbb{Z}}

% cal series
\def\calA{\mathcal{A}}
\def\calB{\mathcal{B}}
\def\calC{\mathcal{C}}
\def\calD{\mathcal{D}}
\def\calE{\mathcal{E}}
\def\calF{\mathcal{F}}
\def\calG{\mathcal{G}}
\def\calH{\mathcal{H}}
\def\calI{\mathcal{I}}
\def\calJ{\mathcal{J}}
\def\calK{\mathcal{K}}
\def\calL{\mathcal{L}}
\def\calM{\mathcal{M}}
\def\calN{\mathcal{N}}
\def\calO{\mathcal{O}}
\def\calP{\mathcal{P}}
\def\calQ{\mathcal{Q}}
\def\calR{\mathcal{R}}
\def\calS{\mathcal{S}}
\def\calT{\mathcal{T}}
\def\calU{\mathcal{U}}
\def\calV{\mathcal{V}}
\def\calW{\mathcal{W}}
\def\calX{\mathcal{X}}
\def\calY{\mathcal{Y}}
\def\calZ{\mathcal{Z}}
%%% custom notation %%%

% vector calculus + partial derivatives 
\def\del{{\partial}}
\newcommand{\deriv}[2][]{\frac{d^{#1}}{d{#2}^{#1}}}
\newcommand{\derivP}[2][]{\frac{\del^{#1}}{\del{#2}^{#1}}}

% linear algebra / matrix notation
\newcommand{\iden}[1]{\mathbb{I}_{{#1} \times {#1}}} % n-by-n identity matrix 
\newcommand{\tpose}[1]{{#1}^{\top}} % matrix transpose

% indicator function
\def\1{{\mathbf 1}}
\newcommand{\indic}[1]{\1_{[#1]}}

% statistics terminology
\newcommand{\expec}[2][]{\bbE_{#1}[#2]} 
\newcommand{\prob}[2][]{\bbP_{#1}(#2)}
\newcommand{\var}[2][]{\text{Var}_{#1}[#2]}
\newcommand{\bias}[1]{\textbf{bias}(#1)}
\newcommand{\stderror}[1]{\textbf{se}(#1)}
\newcommand{\MSE}[1]{\text{MSE}(#1)}
\def\simiid{\sim_{\mbox{\tiny \textrm{iid}}}} %sampled i.i.d

% miscellaneous 
\def\fstar{f^{*}}
\usepackage{graphicx, amssymb, amsmath, amsthm, amsfonts, mathrsfs}
\usepackage{multirow, makeidx}
\usepackage{mathtools}
\usepackage{enumerate, enumitem}
\usepackage{pifont}

\usepackage[ruled, linesnumbered]{algorithm2e}
\SetKwRepeat{Repeat}{repeat}{until} 

\usepackage{multicol}
\setlength{\columnsep}{40pt}

\usepackage{titlesec}
\titleformat{\section}
  {\normalfont\Large\bfseries}
  {}
  {0pt}
  {}
\titleformat{\subsection}
  {\normalfont\large\bfseries}
  {}
  {0pt}
  {}

\usepackage{geometry}
\geometry{
    letterpaper,
    left = 0.75in,
    right = 0.75in,
    top = 1.0in,
    bottom = 1.0in    
}
\usepackage{parskip}
\usepackage{scalefnt}
\usepackage{caption,subcaption}
\usepackage{hyperref}
\hypersetup{
    colorlinks=true,
    linkcolor=cyan,
    filecolor=magenta,      
    urlcolor=blue
}

\pagenumbering{gobble}
\numberwithin{figure}{section}
\renewcommand{\thefigure}{\arabic{section}.\arabic{figure}}
\numberwithin{equation}{section}
\renewcommand{\theequation}{\arabic{section}.\arabic{equation}}

\def\BayesOpt{\texttt{BayesOpt}}
\def\EI{\texttt{EI}}
\def\calGP{\mathcal{GP}}

\newcommand{\bs}[1]{\boldsymbol{#1}}
\def\bsx{\bs{x}}
\def\bell{\bs{\ell}}
\def\xnext{\bsx_{\textrm{next}}}
\def\xast{\bsx_{*}}
\def\rhoast{\bs{\rho}_{*}}

\usepackage{xparse}
\NewDocumentCommand{\xinc}{o}{%
  \bsx_{\textrm{inc}\IfValueT{#1}{,{#1}}}
}
\NewDocumentCommand{\yinc}{o}{%
  y_{\textrm{inc}\IfValueT{#1}{,{#1}}}
}


\begin{document}

\section{Context}

This document contains some equations and formulas based on the theoretical claims in the article \href{https://arxiv.org/abs/2402.02229}{\textit{Vanilla Bayesian Optimization Performs Great in High Dimensions}} by Carl Hvarfner, Erik O. Hellsten, and Luigi Nardi.

In particular, we look at the proof provided in Appendix C.1 of this article under some slightly altered assumptions.

\noindent\rule{\textwidth}{0.8pt}

\section{Theoretical Setting}

We want to perform Bayesian Optimization (\BayesOpt) to find the global maximum of an unknown real-valued function $\fstar(\cdot)$ defined on a compact space $\calX \subset \bbR^{D}$.

To do this, we utilize a Gaussian process surrogate model of the form $\fstar \sim \calGP\left(\mu(\cdot), k_{\bell}(\cdot, \cdot)\right)$, where $k_{\bell}(\cdot, \cdot)$ is a kernel function with respect to some lengthscale hyperparameter $\bell \in \bbR^{D}$.

\section{Notation}

We use $\calD_{t}$ to denote our dataset after $t$ iterations of \BayesOpt, with $\calD_{0}$ representing the initial dataset prior to any iterations of the Bayesian optimization algorithm. 

Additionally, we let $\xinc$ and $\yinc$ represent the \textbf{incumbents} (the largest values observed thus far), with $\xinc[t]$ and $\yinc[t]$ representing the incumbents after $t$ iterations of \BayesOpt. 

\section{Assumptions}

We will make the following assumptions about our Gaussian process model:

\begin{itemize}[label=\ding{228}]

  \item We assume $\calX$ is a \textbf{convex} subset of $\bbR^{D}$.
  \begin{itemize}[label=\ding{118}]
    \item For simplicity, we fix $\calX = [0, 1]^{D}$.
    \item \textbf{Note:} The original paper also fixes $\calX = [0, 1]^{D}$. 
  \end{itemize}

  \item We fix the prior $f(\bsx) \sim \calN(\mu_{f}, \sigma_{f}^{2})$ for all $\bsx \in \calX$.
  \begin{itemize}[label=\ding{118}]
    \item \textbf{Note:} The original paper also makes this assumption, but the authors use $c$ to denote the prior mean instead of $\mu_{f}$.
  \end{itemize}

  \item We assume that the initial dataset $\calD_{0}$ contains one observation, $\left(\bsx_{0}, y_{0}\right)$, where $y_{0} > \mu_{f}$.
  \begin{itemize}[label=\ding{118}]
    \item \textbf{Note:} The original paper assumes the incumbent $y_{\text{max}}$ is greater than the prior mean $c$. 
  \end{itemize}

  \item We query noisy observations $y_{i} = f(\bsx_{i}) + \epsilon_{i}$, with $\epsilon_{i} \simiid \calN(0, \sigma_{\epsilon})^{2}$.
  \begin{itemize}[label=\ding{118}]
    \item \textbf{Note:} The original paper also makes this assumption.
  \end{itemize}

  \item The kernel $k_{\bell}(\cdot, \cdot)$ is a \textbf{bump function} based on a distance metric on $\calX$ which is affected by the lengthscale hyperparameter $\bell$.
  \begin{itemize}[label=\ding{118}]
    \item This bump function has some maximal distance $B \in \bbR^{+}$ for which the kernel $k$ is non-zero. We assume that this maximal distance $B$ is `small' relative to the overall search space $\calX$.
    \item For simplicity, we will fix $\bell = \begin{pmatrix}1 & \cdots & 1 
    \end{pmatrix}^{\top}$.
    \item Thus, $k(x, x') = \begin{cases}\exp\left(-\frac{1}{1 - B^{-2}\lVert x - x' \rVert^{2}}\right) & \lVert x - x' \rVert < B\\ 0 & \textrm{otherwise}\end{cases}$
    \item \textbf{Note:} The original paper makes a much stricter (and less reasonable) assumption, which is that the next queried point is correlated with at most one existing observation.
  \end{itemize}

  \item The next queried point in each iteration of the \BayesOpt{} algorithm is chosen by maximizing expected improvement (\EI{})
  \begin{itemize}[label=\ding{118}]
    \item \textbf{Note:} The original paper also makes this assumption.
  \end{itemize}

\end{itemize}

\section{Bounding Correlation}

Next, we will analyze the existence of the `Boundary Problem' for Bayesian Optimization as in the original paper, under these loosened assumptions. In this paper, the authors proved a lower bound on the correlation $\bs{\rho} = k(\xnext, \xinc)/\sigma_{f}^{2}$ between the next-queried point $y(\xnext)$ and the incumbent $\xinc$, where $$\xnext := \arg\max_{\xast \in \calX}\EI(\xast)$$
We assumed that $\calD_{0} = \left\{(\bsx_{0}, y_{0})\right\}$, and that $y_{0} > \mu_{f}$. Thus, $\xinc[0] = \bsx_{0}$ and $\yinc[0] = y_0 > \mu_{f}$ are the incumbents. For any $\xast \in \calX$, computing the posterior yields the following:
\begin{align*}
\expec{f(\xast) \mid \calD_{0}} &= \mu(\xast) + k(\xast, \bsx_{0})\left(\sigma^{2}_{f} + \sigma^{2}_{\epsilon}\right)^{-1}(y_{0} - \mu(\xast))\\
&= \mu_{f} + k(\xast, \bsx_{0})\left(\sigma^{2}_{f} + \sigma^{2}_{\epsilon}\right)^{-1}(y_{0} - \mu_{f})\\
% \intertext{As $y_0 > \mu_{f}$, we find that $\expec{f(\xast) \mid \calD_{0}} \ge \mu_{f}$, with equality if and only if $k(\xast, \bsx_{0}) = 0$. Note that $k(\xast, \bsx_{0})$ is non-zero if and only if $\lVert \xast - \bsx_{0} \rVert < B$.}
\var{f(\xast) \mid \calD_{0}} &= k(\xast, \xast) - k(\xast, \bsx_{0})\left(\sigma^{2}_{f} + \sigma^{2}_{\epsilon}\right)^{-1}k(\bsx_{0}, \xast)\\
&= \sigma^{2}_{f} - k(\xast, \bsx_{0})\left(\sigma^{2}_{f} + \sigma^{2}_{\epsilon}\right)^{-1}k(\bsx_{0}, \xast)\\
f(\xast) \mid \calD_{0} &\sim \calN\left(\mu_{f} + k(\xast, \bsx_{0})\left(\sigma^{2}_{f} + \sigma^{2}_{\epsilon}\right)^{-1}(y_{0} - \mu_{f}), \sigma^{2}_{f} - k(\xast, \bsx_{0})\left(\sigma^{2}_{f} + \sigma^{2}_{\epsilon}\right)^{-1}k(\bsx_{0}, \xast)\right)
\intertext{Note that for any $\xast \in \calX$ where $\lVert \xast - \bsx_{0}\rVert \ge B$, we have $k(\xast, \bsx_{0}) = 0$, and thus the posterior distribution is $f(\xast) \mid \calD_{0} \sim \calN(\mu_{f}, \sigma^{2}_{f})$, which is identical to the prior.}
\intertext{Rewriting the kernel in terms of the correlation $\rhoast = k(\xast, \bsx_{0})/\sigma^{2}_{f}$ yields the following. \textbf{Note:} Remember, $\rhoast$ is not a constant, it is a function of $\xast$. The $\rhoast$ notation is conventional shorthand for the function $\rho(\xast) = k(\xast, \bsx_{0})/\sigma^{2}_{f}$.}
f(\xast) \mid \calD_{0} &\sim \calN\left(\mu_{f} + \rhoast\frac{\sigma^{2}_{f}}{\sigma^{2}_{f} + \sigma^{2}_{\epsilon}}(y_{0} - \mu_{f}), \sigma^{2}_{f} - \rhoast^{2}\frac{\sigma^{4}_{f}}{\sigma^{2}_{f} + \sigma^{2}_{\epsilon}}\right)
\intertext{For ease of notation, we define $r_{v}^{2} = \frac{\sigma^{2}_{f}}{\sigma^{2}_{f}+\sigma^{2}_{\epsilon}}$ to be the ratio of variances.}
% \intertext{When $\sigma^{2}_{\epsilon} = 0$, this simplifies to $f(\xast) \mid \calD_{0} \sim \calN\left(\mu_{f} + \rhoast(y_{0} - \mu_{f}), \sigma^{2}_{f} - \rhoast^{2}\sigma^{2}_{f}\right)$}
f(\xast) \mid \calD_{0} &\sim \calN\left(\mu_{f}\left(1 - \rhoast r_{v}^{2}\right) + \rhoast r_{v}^{2} y_{0}, \sigma^{2}_{f}\left(1 - \rhoast^{2}r_{v}^{2}\right)\right)
\end{align*}

This gives us a closed form for the posterior mean and variance of $f(\xast)$ which we can can use to compute the expected improvement from querying $\xast$.
\begin{align*}
\EI(\xast) &= \left(\mu(\xast) - y_{0}\right)\Phi\left(\frac{\mu(\xast) - y_{0}}{\sigma(\xast)}\right) + \sigma(\xast)\phi\left(\frac{\mu(\xast) - y_{0}}{\sigma(\xast)}\right)
\intertext{Plugging in $\mu(\xast)$ and $\sigma(\xast)$:}
\EI(\xast) &= (y_{0} - \mu_{f})(\rhoast r_{v}^{2} - 1)\Phi\left(\frac{(y_{0} - \mu_{f})(\rhoast r_{v}^{2} - 1)}{\sigma_{f}\sqrt{1 - \rhoast^{2}r_{v}^{2}}}\right) + \sigma_{f}\sqrt{1 - \rhoast^{2}r_{v}^{2}}\phi\left(\frac{(y_{0} - \mu_{f})(\rhoast r_{v}^{2} - 1)}{\sigma_{f}\sqrt{1 - \rhoast^{2}r_{v}^{2}}}\right)
\intertext{\textbf{Note:} In this setup, $\EI(\xast)$ only depends on $k(\xast, \bsx_{0})$, not on $\xast$ itself.}
\end{align*}
\end{document}