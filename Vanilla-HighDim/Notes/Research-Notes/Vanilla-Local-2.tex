\documentclass[11pt]{article}
%%% full alphabets of different styles %%%

% bf series
\def\bfA{\mathbf{A}}
\def\bfB{\mathbf{B}}
\def\bfC{\mathbf{C}}
\def\bfD{\mathbf{D}}
\def\bfE{\mathbf{E}}
\def\bfF{\mathbf{F}}
\def\bfG{\mathbf{G}}
\def\bfH{\mathbf{H}}
\def\bfI{\mathbf{I}}
\def\bfJ{\mathbf{J}}
\def\bfK{\mathbf{K}}
\def\bfL{\mathbf{L}}
\def\bfM{\mathbf{M}}
\def\bfN{\mathbf{N}}
\def\bfO{\mathbf{O}}
\def\bfP{\mathbf{P}}
\def\bfQ{\mathbf{Q}}
\def\bfR{\mathbf{R}}
\def\bfS{\mathbf{S}}
\def\bfT{\mathbf{T}}
\def\bfU{\mathbf{U}}
\def\bfV{\mathbf{V}}
\def\bfW{\mathbf{W}}
\def\bfX{\mathbf{X}}
\def\bfY{\mathbf{Y}}
\def\bfZ{\mathbf{Z}}
\def\bfx{\mathbf{x}}
\def\bfmu{\boldsymbol{\mu}}
\def\bfSigma{\mathbf{\Sigma}}

% bb series
\def\bbA{\mathbb{A}}
\def\bbB{\mathbb{B}}
\def\bbC{\mathbb{C}}
\def\bbD{\mathbb{D}}
\def\bbE{\mathbb{E}}
\def\bbF{\mathbb{F}}
\def\bbG{\mathbb{G}}
\def\bbH{\mathbb{H}}
\def\bbI{\mathbb{I}}
\def\bbJ{\mathbb{J}}
\def\bbK{\mathbb{K}}
\def\bbL{\mathbb{L}}
\def\bbM{\mathbb{M}}
\def\bbN{\mathbb{N}}
\def\bbO{\mathbb{O}}
\def\bbP{\mathbb{P}}
\def\bbQ{\mathbb{Q}}
\def\bbR{\mathbb{R}}
\def\bbS{\mathbb{S}}
\def\bbT{\mathbb{T}}
\def\bbU{\mathbb{U}}
\def\bbV{\mathbb{V}}
\def\bbW{\mathbb{W}}
\def\bbX{\mathbb{X}}
\def\bbY{\mathbb{Y}}
\def\bbZ{\mathbb{Z}}

% cal series
\def\calA{\mathcal{A}}
\def\calB{\mathcal{B}}
\def\calC{\mathcal{C}}
\def\calD{\mathcal{D}}
\def\calE{\mathcal{E}}
\def\calF{\mathcal{F}}
\def\calG{\mathcal{G}}
\def\calH{\mathcal{H}}
\def\calI{\mathcal{I}}
\def\calJ{\mathcal{J}}
\def\calK{\mathcal{K}}
\def\calL{\mathcal{L}}
\def\calM{\mathcal{M}}
\def\calN{\mathcal{N}}
\def\calO{\mathcal{O}}
\def\calP{\mathcal{P}}
\def\calQ{\mathcal{Q}}
\def\calR{\mathcal{R}}
\def\calS{\mathcal{S}}
\def\calT{\mathcal{T}}
\def\calU{\mathcal{U}}
\def\calV{\mathcal{V}}
\def\calW{\mathcal{W}}
\def\calX{\mathcal{X}}
\def\calY{\mathcal{Y}}
\def\calZ{\mathcal{Z}}
%%% custom notation %%%

% vector calculus + partial derivatives 
\def\del{{\partial}}
\newcommand{\deriv}[2][]{\frac{d^{#1}}{d{#2}^{#1}}}
\newcommand{\derivP}[2][]{\frac{\del^{#1}}{\del{#2}^{#1}}}

% linear algebra / matrix notation
\newcommand{\iden}[1]{\mathbb{I}_{{#1} \times {#1}}} % n-by-n identity matrix 
\newcommand{\tpose}[1]{{#1}^{\top}} % matrix transpose

% indicator function
\def\1{{\mathbf 1}}
\newcommand{\indic}[1]{\1_{[#1]}}

% statistics terminology
\newcommand{\expec}[2][]{\bbE_{#1}[#2]} 
\newcommand{\prob}[2][]{\bbP_{#1}(#2)}
\newcommand{\var}[2][]{\text{Var}_{#1}[#2]}
\newcommand{\bias}[1]{\textbf{bias}(#1)}
\newcommand{\stderror}[1]{\textbf{se}(#1)}
\newcommand{\MSE}[1]{\text{MSE}(#1)}
\def\simiid{\sim_{\mbox{\tiny \textrm{iid}}}} %sampled i.i.d

% miscellaneous 
\def\fstar{f^{*}}
\usepackage{graphicx, amssymb, amsmath, amsthm, amsfonts, mathrsfs}
\usepackage{multirow, makeidx}
\usepackage{mathtools}
\usepackage{enumerate, enumitem}
\usepackage{pifont}

\usepackage[ruled, linesnumbered]{algorithm2e}
\SetKwRepeat{Repeat}{repeat}{until} 

\usepackage{multicol}
\setlength{\columnsep}{40pt}

\usepackage{titlesec}
\titleformat{\section}
  {\normalfont\Large\bfseries}
  {}
  {0pt}
  {}
\titleformat{\subsection}
  {\normalfont\large\bfseries}
  {}
  {0pt}
  {}

\usepackage{geometry}
\geometry{
    letterpaper,
    left = 0.75in,
    right = 0.75in,
    top = 1.0in,
    bottom = 1.0in    
}
\usepackage{parskip}
\usepackage{scalefnt}
\usepackage{caption,subcaption}
\usepackage{hyperref}
\hypersetup{
    colorlinks=true,
    linkcolor=cyan,
    filecolor=magenta,      
    urlcolor=blue
}

\pagenumbering{gobble}
\numberwithin{figure}{section}
\renewcommand{\thefigure}{\arabic{section}.\arabic{figure}}
\numberwithin{equation}{section}
\renewcommand{\theequation}{\arabic{section}.\arabic{equation}}

\def\BayesOpt{\texttt{BayesOpt}}
\def\EI{\texttt{EI}}
\def\calGP{\mathcal{GP}}

\newcommand{\bs}[1]{\boldsymbol{#1}}
\def\bsx{\bs{x}}
\def\bell{\bs{\ell}}
\def\xnext{\bsx_{\textrm{next}}}
\def\xast{\bsx_{*}}
\def\rhoast{\bs{\rho}_{*}}

\usepackage{xparse}
\NewDocumentCommand{\xinc}{o}{%
  \bsx_{\textrm{inc}\IfValueT{#1}{,{#1}}}
}
\NewDocumentCommand{\yinc}{o}{%
  y_{\textrm{inc}\IfValueT{#1}{,{#1}}}
}


\begin{document}

\section{Context}

This document contains mathematical derivations for an altered version of a \BayesOpt{} algorithm in which the dataset only ever has two recorded values, with additional values `thrown out' at each iteration of the algorithm.

Part of the reason for creating this algorithm is to mathematically determine whether the algorithm proposed in \href{https://arxiv.org/abs/2402.02229}{\textit{Vanilla Bayesian Optimization Performs Great in High Dimensions}} is primarily (from a conceptual standpoint) just performing local Bayesian optimization in a high-dimensional space.

\noindent\rule{\textwidth}{0.8pt}

\section{Theoretical Setup}

We want to perform Bayesian Optimization (\BayesOpt) to find the global maximum of an unknown real-valued function $\fstar(\cdot)$ defined on a compact space $\calX \subset \bbR^{D}$. For the sake of simplicity, we will assume that $\calX = [0,1]^{D}$.

To achieve this goal, we use a Gaussian process model $\fstar \sim \calGP\left(\mu_{f}(\cdot), \sigma_{f}^{2}k_{\bell}(\cdot, \cdot)\right)$, where $k_{\bell}(\cdot, \cdot)$ is an RBF (radial basis function) kernel with lengthscale hyperparameter $\bell \in \bbR^{D}$. For now, we fix $\bell = \begin{pmatrix}1 & \cdots & 1\end{pmatrix}^{\top}$.

Unlike traditional \BayesOpt{} algorithms which update the posterior of $\fstar$ after each iteration based on \textbf{all} collected datapoints, we propose a `forgetful' algorithm, in which the dataset only ever contains two observations: the \textit{incumbent} (maximum value observed thus far) and the most-recent non-incumbent.

\section{Assumptions}

In the analysis of this algorithm, we make the following assumptions:

\begin{itemize}[label=\ding{228}]

  \item We can model our unknown function as a Gaussian process of the form $\fstar \sim \calGP\left(\mu_{f}(\cdot), \sigma_{f}^{2}k_{\bell}(\cdot, \cdot)\right)$
  \begin{itemize}[label=\ding{118}]
    \item Furthermore, we assume that $\mu_{f}(\bsx) = 0$ for all $\bsx \in \calX$, and that $\sigma_{f} = 1$ by scaling the function as required.
  \end{itemize}

  \item We assume that $\calX$ is a \textbf{convex} subset of $\bbR^{D}$. 
  \begin{itemize}[label=\ding{118}]
    \item Furthermore, we will assume that $\calX$ is the unit hypercube $[0, 1]^{D}$.
  \end{itemize}

  \item We query \textit{noiseless} observations $y_{i} = \fstar(\bsx_{i})$, i.e. the observation noise parameter $\sigma_{\epsilon}$ is 0. 

  \item $k_{\bell}(\cdot, \cdot)$ is an RBF kernel which is affected by the lengthscale hyperparameter $\bell$.
  \begin{itemize}[label=\ding{118}]
    \item For simplicity, we will fix $\bell = \begin{pmatrix}1 & \cdots & 1 \end{pmatrix}^{\top}$.
  \end{itemize}

  \item The next queried point in each iteration of the algorithm is chosen by maximizing the expected improvement (\EI{}) with respect to the current dataset. 

  \item At each iteration $t$ of the algorithm, $y_{0} \ge y_{1}$. This can be done without loss of generality by relabelling as needed.
\end{itemize}

\end{document}