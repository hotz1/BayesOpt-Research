\documentclass[11pt]{article}
%%% full alphabets of different styles %%%

% bf series
\def\bfA{\mathbf{A}}
\def\bfB{\mathbf{B}}
\def\bfC{\mathbf{C}}
\def\bfD{\mathbf{D}}
\def\bfE{\mathbf{E}}
\def\bfF{\mathbf{F}}
\def\bfG{\mathbf{G}}
\def\bfH{\mathbf{H}}
\def\bfI{\mathbf{I}}
\def\bfJ{\mathbf{J}}
\def\bfK{\mathbf{K}}
\def\bfL{\mathbf{L}}
\def\bfM{\mathbf{M}}
\def\bfN{\mathbf{N}}
\def\bfO{\mathbf{O}}
\def\bfP{\mathbf{P}}
\def\bfQ{\mathbf{Q}}
\def\bfR{\mathbf{R}}
\def\bfS{\mathbf{S}}
\def\bfT{\mathbf{T}}
\def\bfU{\mathbf{U}}
\def\bfV{\mathbf{V}}
\def\bfW{\mathbf{W}}
\def\bfX{\mathbf{X}}
\def\bfY{\mathbf{Y}}
\def\bfZ{\mathbf{Z}}
\def\bfx{\mathbf{x}}
\def\bfmu{\boldsymbol{\mu}}
\def\bfSigma{\mathbf{\Sigma}}

% bb series
\def\bbA{\mathbb{A}}
\def\bbB{\mathbb{B}}
\def\bbC{\mathbb{C}}
\def\bbD{\mathbb{D}}
\def\bbE{\mathbb{E}}
\def\bbF{\mathbb{F}}
\def\bbG{\mathbb{G}}
\def\bbH{\mathbb{H}}
\def\bbI{\mathbb{I}}
\def\bbJ{\mathbb{J}}
\def\bbK{\mathbb{K}}
\def\bbL{\mathbb{L}}
\def\bbM{\mathbb{M}}
\def\bbN{\mathbb{N}}
\def\bbO{\mathbb{O}}
\def\bbP{\mathbb{P}}
\def\bbQ{\mathbb{Q}}
\def\bbR{\mathbb{R}}
\def\bbS{\mathbb{S}}
\def\bbT{\mathbb{T}}
\def\bbU{\mathbb{U}}
\def\bbV{\mathbb{V}}
\def\bbW{\mathbb{W}}
\def\bbX{\mathbb{X}}
\def\bbY{\mathbb{Y}}
\def\bbZ{\mathbb{Z}}

% cal series
\def\calA{\mathcal{A}}
\def\calB{\mathcal{B}}
\def\calC{\mathcal{C}}
\def\calD{\mathcal{D}}
\def\calE{\mathcal{E}}
\def\calF{\mathcal{F}}
\def\calG{\mathcal{G}}
\def\calH{\mathcal{H}}
\def\calI{\mathcal{I}}
\def\calJ{\mathcal{J}}
\def\calK{\mathcal{K}}
\def\calL{\mathcal{L}}
\def\calM{\mathcal{M}}
\def\calN{\mathcal{N}}
\def\calO{\mathcal{O}}
\def\calP{\mathcal{P}}
\def\calQ{\mathcal{Q}}
\def\calR{\mathcal{R}}
\def\calS{\mathcal{S}}
\def\calT{\mathcal{T}}
\def\calU{\mathcal{U}}
\def\calV{\mathcal{V}}
\def\calW{\mathcal{W}}
\def\calX{\mathcal{X}}
\def\calY{\mathcal{Y}}
\def\calZ{\mathcal{Z}}
%%% custom notation %%%

% vector calculus + partial derivatives 
\def\del{{\partial}}
\newcommand{\deriv}[2][]{\frac{d^{#1}}{d{#2}^{#1}}}
\newcommand{\derivP}[2][]{\frac{\del^{#1}}{\del{#2}^{#1}}}

% linear algebra / matrix notation
\newcommand{\iden}[1]{\mathbb{I}_{{#1} \times {#1}}} % n-by-n identity matrix 
\newcommand{\tpose}[1]{{#1}^{\top}} % matrix transpose

% indicator function
\def\1{{\mathbf 1}}
\newcommand{\indic}[1]{\1_{[#1]}}

% statistics terminology
\newcommand{\expec}[2][]{\bbE_{#1}[#2]} 
\newcommand{\prob}[2][]{\bbP_{#1}(#2)}
\newcommand{\var}[2][]{\text{Var}_{#1}[#2]}
\newcommand{\bias}[1]{\textbf{bias}(#1)}
\newcommand{\stderror}[1]{\textbf{se}(#1)}
\newcommand{\MSE}[1]{\text{MSE}(#1)}
\def\simiid{\sim_{\mbox{\tiny \textrm{iid}}}} %sampled i.i.d

% miscellaneous 
\def\fstar{f^{*}}
\usepackage{graphicx, amssymb, amsmath, amsthm, amsfonts, mathrsfs}
\usepackage{multirow, makeidx}
\usepackage{mathtools}
\usepackage{enumerate, enumitem}
\usepackage{pifont}

\usepackage[ruled, linesnumbered]{algorithm2e}
\SetKwRepeat{Repeat}{repeat}{until} 

\usepackage{multicol}
\setlength{\columnsep}{40pt}

\usepackage{titlesec}
\titleformat{\section}
  {\normalfont\Large\bfseries}
  {}
  {0pt}
  {}
\titleformat{\subsection}
  {\normalfont\large\bfseries}
  {}
  {0pt}
  {}

\usepackage{geometry}
\geometry{
    letterpaper,
    left = 0.75in,
    right = 0.75in,
    top = 1.0in,
    bottom = 1.0in    
}
\usepackage{parskip}
\usepackage{scalefnt}
\usepackage{caption,subcaption}
\usepackage{hyperref}
\hypersetup{
    colorlinks=true,
    linkcolor=cyan,
    filecolor=magenta,      
    urlcolor=blue
}

% \usepackage{titlesec}
% \titleformat{\section}
%   {\normalfont\Large\bfseries}
%   {}
%   {0pt}
%   {}
  
% \titleformat{\subsection}
%   {\normalfont\large\bfseries}
%   {}                        
%   {0pt}                    
%   {}

\def\BayesOpt{\texttt{BayesOpt}}
\def\EI{\texttt{EI}}
\def\calGP{\mathcal{GP}}
\def\Matern{\textrm{Mat\'{e}rn}}
\pagenumbering{gobble}

\begin{document}
 
\section{Linear Algebra}
\begin{itemize}[label = \ding{228}, itemsep = -3pt, topsep = -10pt, leftmargin = *]
  \item 
  \item 
\end{itemize}

\section{Normal Distributions}
\begin{itemize}[label = \ding{228}, itemsep = -3pt, topsep = -10pt, leftmargin = *]

  \item 
  \underline{Univariate Normal}: $X \sim \calN(\mu, \sigma^2)$ has density $f_{X}(x; \mu, \sigma^2) = (2\pi\sigma^{2})^{-1/2}\exp(-\frac{(x-\mu)^2}{2\sigma^2})$
  \begin{itemize}[label = \ding{118}, itemsep = -2pt, topsep = -10pt]
    \item $X \sim \calN(\mu, \sigma^2) \implies \expec{X} = \mu$ 
    \item $X \sim \calN(\mu, \sigma^2) \implies \var{X} = \sigma^2$ 
    \item
    Any linear combination of \textbf{independent} Normal random variables is also Normal. Example:\\ $X_{1} \sim \calN(\mu_{1}, \sigma^{2}_{1}), X_{2} \sim \calN(\mu_{2}, \sigma^{2}_{2}) \implies aX_{1} + bX_{2} \sim \calN(a\mu_{1} + b\mu_{2}, a^{2}\sigma_{1}^{2} + b^2\sigma_{2}^{2})$
  \end{itemize}
  
  \item 
  \underline{Multivariate Normal}: $\bfX \sim \calN(\bfmu, \bfSigma)$, where $\bfmu \in \bbR^{D}$ and $\bfSigma \in \bbR^{D \times D}$ is positive definite has density $f_{X}(\bfx; \bfmu, \bfSigma) = (2\pi)^{-D/2}\det(\bfSigma)^{-1/2}\exp\big(-\frac{1}{2}(\bfx - \bfmu)^{\top}\bfSigma^{-1}(\bfx - \bfmu)\big)$
  \begin{itemize}[label = \ding{118}, itemsep = -2pt, topsep = -10pt]
    \item $\bfX \sim \calN(\bfmu, \bfSigma) \implies \expec{\bfX} = \bfmu$ 
    \item $\bfX \sim \calN(\bfmu, \bfSigma) \implies \var{\bfX} = \bfSigma$ 
    \item
    If $\bfX$ follows a multivariate Normal distribution, then any linear transformation of $\bfX$ also follows a multivariate Normal. Example: $\bfX \sim \calN(\bfmu, \bfSigma) \implies A\bfX \sim \calN(A\bfmu, A\bfSigma{A^{\top}})$
  \end{itemize}

\end{itemize}

\section{Gaussian Process}
\begin{itemize}[label = \ding{228}, itemsep = -3pt, topsep = -10pt, leftmargin = *]
  \item A Gaussian Process (GP) is a stochastic process which can represent a probability distribution over a function space, e.g. the space of real-valued continuous functions on $[0, 1]$.
  \item A GP can be fully characterized by its mean ($\mu(\cdot)$) and a covariance function $k(\cdot, \cdot)$, also known as a kernel function.

  \item The kernel represents the correlation between the values of a function drawn from a GP as a function of the inputs, which may depend on additional hyperparameters. Common GP kernels include the following:
  \begin{itemize}[label = \ding{118}, itemsep = -2pt, topsep = -10pt]
    \item Linear: $k_{\ell}(x, x') = \ell{x^{\top}x'}$
    \item \href{https://en.wikipedia.org/wiki/Mat%C3%A9rn_covariance_function}{\Matern{} kernel} with parameter $\nu$, which can be simplified for $\nu = n + 1/2, n \in \bbN$. A GP with a $\Matern(\nu)$ kernel is mean-square differentiable $\lceil \nu \rceil - 1$ times. 
    \item Squared Exponential: $k_{\ell}(x, x') = \exp\left(-\frac{\lVert x - x' \rVert^{2}}{2\ell^2}\right)$
    \item Bump function: $k_{\ell}(x, x') = \begin{cases}\exp\left(-\frac{1}{1 - \ell^{-2}\lVert x - x' \rVert^{2}}\right) & \lVert x - x' \rVert < \ell\\ 0 & \textrm{otherwise}\end{cases}$
  \end{itemize}
  
\end{itemize}

\section{Bayesian Optimization}
\begin{itemize}[label = \ding{228}, itemsep = -3pt, topsep = -10pt, leftmargin = *]
  \item Bayesian Optimization (\BayesOpt{}) is a method for finding the global optimum (typically global maximum) of an unknown real-valued function $\fstar(\cdot)$ defined on a compact space $\calX \subset \bbR^{D}$.
  \item This can be accomplished using a Bayesian approach with a Gaussian Process as a surrogate model over the space of functions from $\calX$ to $\bbR$. 
  \item We place a GP prior on $\fstar$ of the form $\fstar \sim \calGP(\mu(\cdot), k(\cdot, \cdot))$ and update this based on observed data.
\end{itemize}

% \section{New Section}
% \begin{itemize}[label = \ding{228}, itemsep = -3pt, topsep = -10pt, leftmargin = *]

%   \item 
%   \begin{itemize}[label = \ding{118}, itemsep = -2pt, topsep = -10pt]
%     \item 
%   \end{itemize}

% \end{itemize}

\end{document}