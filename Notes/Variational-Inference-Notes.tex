\documentclass[11pt]{article}
%%% full alphabets of different styles %%%

% bf series
\def\bfA{\mathbf{A}}
\def\bfB{\mathbf{B}}
\def\bfC{\mathbf{C}}
\def\bfD{\mathbf{D}}
\def\bfE{\mathbf{E}}
\def\bfF{\mathbf{F}}
\def\bfG{\mathbf{G}}
\def\bfH{\mathbf{H}}
\def\bfI{\mathbf{I}}
\def\bfJ{\mathbf{J}}
\def\bfK{\mathbf{K}}
\def\bfL{\mathbf{L}}
\def\bfM{\mathbf{M}}
\def\bfN{\mathbf{N}}
\def\bfO{\mathbf{O}}
\def\bfP{\mathbf{P}}
\def\bfQ{\mathbf{Q}}
\def\bfR{\mathbf{R}}
\def\bfS{\mathbf{S}}
\def\bfT{\mathbf{T}}
\def\bfU{\mathbf{U}}
\def\bfV{\mathbf{V}}
\def\bfW{\mathbf{W}}
\def\bfX{\mathbf{X}}
\def\bfY{\mathbf{Y}}
\def\bfZ{\mathbf{Z}}
\def\bfx{\mathbf{x}}
\def\bfmu{\boldsymbol{\mu}}
\def\bfSigma{\mathbf{\Sigma}}

% bb series
\def\bbA{\mathbb{A}}
\def\bbB{\mathbb{B}}
\def\bbC{\mathbb{C}}
\def\bbD{\mathbb{D}}
\def\bbE{\mathbb{E}}
\def\bbF{\mathbb{F}}
\def\bbG{\mathbb{G}}
\def\bbH{\mathbb{H}}
\def\bbI{\mathbb{I}}
\def\bbJ{\mathbb{J}}
\def\bbK{\mathbb{K}}
\def\bbL{\mathbb{L}}
\def\bbM{\mathbb{M}}
\def\bbN{\mathbb{N}}
\def\bbO{\mathbb{O}}
\def\bbP{\mathbb{P}}
\def\bbQ{\mathbb{Q}}
\def\bbR{\mathbb{R}}
\def\bbS{\mathbb{S}}
\def\bbT{\mathbb{T}}
\def\bbU{\mathbb{U}}
\def\bbV{\mathbb{V}}
\def\bbW{\mathbb{W}}
\def\bbX{\mathbb{X}}
\def\bbY{\mathbb{Y}}
\def\bbZ{\mathbb{Z}}

% cal series
\def\calA{\mathcal{A}}
\def\calB{\mathcal{B}}
\def\calC{\mathcal{C}}
\def\calD{\mathcal{D}}
\def\calE{\mathcal{E}}
\def\calF{\mathcal{F}}
\def\calG{\mathcal{G}}
\def\calH{\mathcal{H}}
\def\calI{\mathcal{I}}
\def\calJ{\mathcal{J}}
\def\calK{\mathcal{K}}
\def\calL{\mathcal{L}}
\def\calM{\mathcal{M}}
\def\calN{\mathcal{N}}
\def\calO{\mathcal{O}}
\def\calP{\mathcal{P}}
\def\calQ{\mathcal{Q}}
\def\calR{\mathcal{R}}
\def\calS{\mathcal{S}}
\def\calT{\mathcal{T}}
\def\calU{\mathcal{U}}
\def\calV{\mathcal{V}}
\def\calW{\mathcal{W}}
\def\calX{\mathcal{X}}
\def\calY{\mathcal{Y}}
\def\calZ{\mathcal{Z}}
%%% custom notation %%%

% vector calculus + partial derivatives 
\def\del{{\partial}}
\newcommand{\deriv}[2][]{\frac{d^{#1}}{d{#2}^{#1}}}
\newcommand{\derivP}[2][]{\frac{\del^{#1}}{\del{#2}^{#1}}}

% linear algebra / matrix notation
\newcommand{\iden}[1]{\mathbb{I}_{{#1} \times {#1}}} % n-by-n identity matrix 
\newcommand{\tpose}[1]{{#1}^{\top}} % matrix transpose

% indicator function
\def\1{{\mathbf 1}}
\newcommand{\indic}[1]{\1_{[#1]}}

% statistics terminology
\newcommand{\expec}[2][]{\bbE_{#1}[#2]} 
\newcommand{\prob}[2][]{\bbP_{#1}(#2)}
\newcommand{\var}[2][]{\text{Var}_{#1}[#2]}
\newcommand{\bias}[1]{\textbf{bias}(#1)}
\newcommand{\stderror}[1]{\textbf{se}(#1)}
\newcommand{\MSE}[1]{\text{MSE}(#1)}
\def\simiid{\sim_{\mbox{\tiny \textrm{iid}}}} %sampled i.i.d

% miscellaneous 
\def\fstar{f^{*}}
\newcommand{\KLdiv}[2]{D_{\textrm{KL}}({#1}\Vert{#2})}
\usepackage{graphicx, amssymb, amsmath, amsthm, amsfonts, mathrsfs}
\usepackage{multirow, makeidx}
\usepackage{mathtools}
\usepackage{enumerate, enumitem}

\usepackage[ruled, linesnumbered]{algorithm2e}
\SetKwRepeat{Repeat}{repeat}{until} 

\usepackage{multicol}
\setlength{\columnsep}{40pt}

\usepackage{titlesec}
\titleformat{\section}
  {\normalfont\Large\bfseries}
  {}
  {0pt}
  {}
\titleformat{\subsection}
  {\normalfont\large\bfseries}
  {}
  {0pt}
  {}

\usepackage{geometry}
\geometry{
    letterpaper,
    left = 0.75in,
    right = 0.75in,
    top = 1.0in,
    bottom = 1.0in    
}
\usepackage{parskip}
\usepackage{scalefnt}
\usepackage{caption,subcaption}
\usepackage{hyperref}
\hypersetup{
    colorlinks=true,
    linkcolor=cyan,
    filecolor=magenta,      
    urlcolor=blue
}

\usepackage{titlesec}
\titleformat{\section}
  {\normalfont\Large\bfseries}
  {}
  {0pt}
  {}
  
\titleformat{\subsection}
  {\normalfont\large\bfseries}
  {}                        
  {0pt}                    
  {}

\pagenumbering{gobble}
\numberwithin{figure}{section}
\renewcommand{\thefigure}{\arabic{section}.\arabic{figure}}
\numberwithin{equation}{section}
\renewcommand{\theequation}{\arabic{section}.\arabic{equation}}

\begin{document}

\section{Variational Inference}

Variational Inference (VI) is a method for approximating a conditional posterior distribution over latent/hidden variables in a Bayesian setting. This is a useful tool, as the resulting posterior distributions can often become computationally complex or entirely intractable. 

\subsection{General Setup}
We assume that ${x}_{1:n} = \{x_{1}, x_{2}, \dots, x_{n}\}$ are observations, with hidden variables ${z}_{1:m} = \{z_1, \dots, z_m\}$ and additional fixed (\textit{hyper}-)parameters $\alpha$. 

We are interested in inference on the hidden variables ${z}_{1:m}$, which invokes a posterior conditional distribution of the form 
\begin{equation}
    p(z_{1:m} \mid x_{1:n}, \alpha) = \frac{p(z_{1:m}, x_{1:n} \mid \alpha)}{p(x_{1:n} \mid \alpha)} = \frac{p(z_{1:m}, x_{1:n} \mid \alpha)}{\int_{z}p(z_{1:m}, x_{1:n} \mid \alpha)dz}
\end{equation}

The denominator for this posterior distribution is often difficult to compute, if not fully intractable, so we must approximate the distribution $\prob{z_{1:m}\mid x_{1:n}, \alpha}$. One approach is to consider a \textbf{variational family} of distributions $\calQ = \{q(z_{1:m} \mid \nu)\}$ over the latent variables $z_{1:m}$, and finding the distribution in the family which is the most suitable (i.e. closest) proxy for the `true' posterior distribution $p(z_{1:m} \mid x_{1:n}, \alpha)$.

\section{Kullback-Leibler Divergence}
To measure the `closeness' of two probability distributions $P$ and $Q$ defined on the same space, we can use the \textbf{Kullback-Leibler} (KL) divergence. This divergence is defined as 
\begin{equation}
    \KLdiv{P}{Q} := \int P(x)\log\left(\frac{P(x)}{Q(x)}\right)\textrm{d}P = \bbE_{P}\left[\log\left(\frac{P(x)}{Q(x)}\right)\right]
\end{equation}
Note that this is not a distance metric, as $\KLdiv{P}{Q} \ne \KLdiv{Q}{P}$. To get a distribution in our variational family which is close to the true posterior, we aim to have a low KL divergence.

\section{Evidence Lower Bound}
We define the \textbf{Evidence Lower Bound} (ELBO) as a function of our distribution which we can minimize for choosing the member of the variational family $q(z_{1:m} \mid \nu)$. For probability distributions $P, Q$, we have the following: 
\begin{align*}
    \log\left(P(x)\right) &= \log\left(\int P(x, z)\textrm{d}z\right) \tag{Marginal distribution}\\
    &= \log\left(\int P(x, z)\frac{Q(z)}{Q(z)}\textrm{d}z\right)\\
    &= \log\left(\int Q(z)\left[\frac{P(x,z)}{Q(z)}\right]\textrm{d}z\right)\\
    &= \log\left(\bbE_{Q}\left[\frac{P(x, Z)}{Q(Z)}\right]\right)\\
    &\ge \bbE_{Q}\left[\log\left(\frac{P(x, Z)}{Q(Z)}\right)\right] \tag{Jensen's Inequality}
\end{align*}
We define the ELBO as $\bbE_{Q}\left[\log\left(\frac{P(x, Z)}{Q(Z)}\right)\right] = \bbE_{Q}\left[\log\left({P(x, Z)}\right)\right] - \bbE_{Q}\left[\log\left({Q(Z)}\right)\right]$. Note that $-\KLdiv{Q}{P} = \bbE_{Q}\left[\log\left(\frac{P(x, Z)}{Q(Z)}\right)\right]$, so the ELBO is the negative KL divergence. Finding a distribution $Q(z) \in \calQ$ which maximizes the ELBO yields the tightest possible bound on the marginal probability $\log(P(x))$. 

Additionally, for some marginal distribution $p(z \mid x)$ and some ``variational'' distribution $q(z) \in \calQ$ we have the following result:
\begin{align*}
    \KLdiv{q(z)}{p(z \mid x)} &= \bbE_{q}\left[\log\left(\frac{q(Z)}{p(Z \mid x)}\right)\right]\\
    &= \bbE_{q}\left[\log\left(\frac{q(Z)}{p(x, Z)/p(x)}\right)\right]\\
    &= \bbE_{q}\left[\log\left({q(Z)}\right)\right] - \bbE_{q}\left[\log\left({p(x, Z)}\right)\right] + \bbE_{q}\left[\log\left({p(x)}\right)\right]\\
    &= \log(p(x)) - \bbE_{q}\left[\log\left(\frac{p(x, Z)}{q(Z)}\right)\right] \tag{$\log(p(x)) - \textrm{ELBO}$}\\
    &= \log(p(x)) + \KLdiv{q(z)}{p(x, z)} \tag{Alternative formulation}
\end{align*}
Thus, the KL divergence between the ``variational'' distribution $q(z) \in \calQ$ and the marginal distribution $p(z \mid x)$ is the difference between the log-marginal distribution and the ELBO, which is the Jensen gap.

As $\log(p(x))$ is constant, we see that maximizing the ELBO is equivalent to minimizing the KL divergence between the conditional posterior and variational distribution. 

\section{EULBO}
For Bayesian Optimization, a variational inference approach can be helpful as a means for approximation since exact Bayesian Optimization via a Gaussian Process requires $\calO(n^3)$ runtime.

One potential issue with the use of VI in this setting is that the `traditional' variational inference setup requires choosing a distribution $q(z) \in \calQ$ which maximizes the ELBO. However, this is not ideal for BayesOpt, as the goal for BayesOpt is to simply find the global maximum of some unknown function $\fstar$, not to get a good global approximation of $\fstar$. 
\end{document}